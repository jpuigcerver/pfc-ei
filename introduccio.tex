\chapter{Introducció}
\label{cap:int}

\section{Motivació}
Des de fa unes poques dècades, els ordinadors permeten emmagatzemar una àmplia varietat d'informació de manera fiable i replicada per a que puga sobreviure al pas del temps, mantenir-la organitzada per a que siga fàcilment utilitzable i fer-la accessible arreu del món i quasi universalment. Però durant segles, l'única forma de transmetre el coneixement i emmagatzemar-lo de manera més o menys segura ha sigut mitjançat l'escriptura. Precisament el fet de mantenir el coneixement en llibres, manuscrits primer i impresos després, que permeten la seva preservació i més fàcil difusió, ha sigut una de les principals bases de tot el desenvolupament del coneixement humà, especialment científic i tecnològic, arreu del món i és el principal motor pel qual avui gaudim d'unes millors condicions de vida que les dels nostres avantpassats.

Malauradament, els llibres manuscrits i impresos no sempre han tingut èxit en la seva missió de preservar el coneixement. Sols cal recordar el desastrós incendi de l'antiga Biblioteca d'Alexandria que provocà que milers d'obres d'autors de l'antiguitat es perderen per sempre. Ajudar a la preservació de la informació continguda en els llibres manuscrits i ajudar també a la cerca del contingut en aquests, són dos dels motius que van fer nàixer el Reconeixement de Text Manuscrit (HTR, de l'anglès \emph{Handwriting Text Recognition}) a principis del segle XX.

A pesar de tot el progrés aconseguit en els últims anys pel Reconeixement de Text Manuscrit, aquest té encara molts problemes per resoldre causats perquè gran part de la variabilitat que es troba en les imatges que s'utilitzen per reconèixer el text dels llibres no aporta cap informació rellevant per a la classificació dels símbols representats i dificulta el seu reconeixement. Per exemple, un mateix autor no escriu un mateix símbol sempre de la mateixa forma, ni de la mateixa grandària i ni tan sols amb la mateixa orientació; i l'objectiu és que tots aquests diferents traços siguen classificats de la mateixa manera. Per això, un dels components fonamentals de qualsevol sistema de reconeixement de l'escriptura és la normalització d'aquesta imatge, un procés que tracta de reduir aquesta variabilitat.

Aquest projecte compara i quantifica les diferències entre dues alternatives per solucionar un dels problemes que forma part d'aquest procés de normalització: la segmentació del del cos central del text manuscrit. El cos central d'una línia de text manuscrit és aquella porció de la línia on resideix el cos central de cadascun dels símbols que formen el text. Les dues alternatives estudiades per a aquesta segmentació del cos central estan basades en un enfoc heurístic del problema, on un algorisme amb unes regles pre-establertes determina quina és la regió del cos central, i una altra basada en tècniques d'aprenentatge supervisat, on un humà ha segmentat manualment el cos central d'un conjunt d'imatges de mostra i ha entrenat el sistema per a que intente segmentar de manera semblant les noves imatges. Els detalls es veuran en el capítol \ref{cap:seg}.


\section{Reconeixement de formes}

\section{Reconeixement de text manuscrit}
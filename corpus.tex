\chapter{Corpus}
\label{cap:corpus}
En aquest capítol es detallen les dades utilitzades per a realitzar els experiments que quantifiquen les diferències entre els dos mètodes descrits en el capítol \ref{cap:segmentacio}.

\section{IAMDB}\label{sec:corpus_iamdb}
El corpus IAMDB\footnote{\url{http://www.iam.unibe.ch/fki/databases/iam-handwriting-database}} fou recopilat pel grup d'investigació \emph{Computer Vision and Artificial Intelligence} (FKI) dins del \emph{Institute of Computer Science an Applied Mathematics} (IAM), a l'Universitat de Berna. El corpus és d'accés gratuit per a propòsits de recerca i és un dels més utilitzats per al reconeixement de text manuscrit. La primera versió es presentà en la ICDAR (International Conference of Document Analysis and Recognition) el 1999 \cite{MB99}. El corpus és una transcripció manual del corpus Lancaster-Oslo-Bergen (LOB), descrit en la secció \ref{sec:corpus_lob}. Diferents paràgrafs del LOB es repartiren a un grup de persones que reescriviren el text manualment sense cap tipus de restricció en quan al tipus o estil d'escriptura. En 2002, el text fou segmentat per línies i per paraules aïllades \cite{ZB02} i presentada l'última versió en la revista IJDAR \cite{MB02}. Els experiments realitzats en aquest treball utilitzaren la versió del corpus segmentada per línies.\\

La taula \ref{tab:iamdb_original} conté les estadístiques de la versió de 2002 del corpus IAMDB, que ha sigut utilitzada. El corpus original compta amb dos conjunts de validació. Per a aquest projecte, a l'igual que han fet altres autors \cite{bertolami2008ensemble, graves2009novel, espana2011improving}, s'ha utilitzat part del primer conjunt de validació per a ampliar els conjunts de test i el segon de validació, de manera que la distribució final utilitzada per a l'entrenament, validació i test dels sistemes ha sigut l'expresada en la taula \ref{tab:iamdb_used}. Aquests conjunts són totalment disjunts i on cada escriptor ha participat únicament en un dels conjunts. El corpus IAMDB fou utilitzat per a entrenar els models morfològics del reconeixedor de text.\\

\begin{table}
\centering
\begin{tabular}{|c|c|c|}
\hline
Conjunt & Línies & Escriptors \\
\hline
Train & 6161 & 283 \\
Validation 1 & 900 & 46 \\
Validation 2	 & 940 & 43 \\
Test & 1861 & 128\\
\hline
Total & 9862 & 500\\
\hline
\end{tabular}
\caption{Estadístiques del corpus IAMDB original.}\label{tab:iamdb_original}
\end{table}

\begin{table}
\centering
\begin{tabular}{|c|c|c|}
\hline
Conjunt & Línies & Escriptors \\
\hline
Train & 6161 & 283 \\
Validation & 920 & 56 \\
Test & 2781 & 161\\
\hline
Total & 9862 & 500\\
\hline
\end{tabular}
\caption{Estadístiques de la partició feta a partir del corpus IAMDB original.}\label{tab:iamdb_used}
\end{table}

La figura \ref{fig:iamdb_examples} conté algunes de les imatges que formen part de la versió corpus IAMDB utilitzada.
\begin{figure}
\centering
\begin{subfigure}[b]{0.8\textwidth}
\centering
\includegraphics[width=0.6\textwidth]{images/pending_image.eps}
\end{subfigure}\\
\begin{subfigure}[b]{0.8\textwidth}
\centering
\includegraphics[width=0.6\textwidth]{images/pending_image.eps}
\end{subfigure}\\
\begin{subfigure}[b]{0.8\textwidth}
\centering
\includegraphics[width=0.6\textwidth]{images/pending_image.eps}
\end{subfigure}\\
\caption{Exemples de línies de text extretes del corpus IAMDB.}\label{fig:iamdb_examples}
\end{figure}

% ======================================================================
% CORPUS BROWN
\section{Brown}\label{sec:corpus_brown}
El Corpus Estàndard d'Anglès Americà del Present, o simplement corpus Brown, va ser presentat originalment el 1961\cite{francis1979brown} per la Universitat de Brown i conté 500 textos d'aproximadament 2000 paraules cadascun, escrites en anglès americà actual i un total de 1.014.312 paraules distintes entre tots els textos. El corpus conté informació sobre les categories de cada paraula en els textos, però aquesta informació no fou utilitzada per al desenvolupament d'aquest projecte. La taula \ref{tab:corpus_lm} conté un resum dels textos continguts en el corpus Brown. Aquest corpus fou utilitzat per a construir el model del llenguatge utilitzat en els experiments.

% ======================================================================
% CORPUS LOB
\section{Lancaster-Oslo-Bergen}\label{sec:corpus_lob}
El corpus Lancaster-Oslo-Bergen (LOB) fou recopilat i presentat el 1986\cite{johansson1986tagged} per investigadors de la Universitat de Lancaster, la Universitat d'Oslo i el \emph{Norwegian Computing Centre for the Humanities}, en Bergen. El corpus fou recopilat com una alternativa en anglès britànic al corpus de la Universitat de Brown, descrit en la secció \ref{sec:corpus_brown}, que fou desenvolupat a partir de textos en anglès americà. El LOB conté 500 textos d'unes 2000 paraules aproximadament i aproximadament un milió de paraules distintes en tot el corpus. Cada paraula del corpus fou anotada posteriorment en categories, encara que aquesta informació no ha sigut utilitzada per al desenvolupament del projecte. La taula \ref{tab:corpus_lm} conté un resum dels textos continguts en el corpus LOB. Part d'aquest corpus fou utilitzat per a construir el model del llenguatge dels experiments. S'exclogueren aquells texts que s'havien utilitzat per transcriure les línies de text de \emph{test} del corpus IAMDB.

% ======================================================================
% CORPUS WELLINGTON
\section{Wellington}\label{sec:corpus_wellington}
El corpus Wellington d'Anglès escrit de Nova Zelanda, o corpus Wellington, fou presentat el 1993\cite{bauer1993manual} per la Universitat Victoria de Wellington, a Nova Zelanda, i fou desenvolupat a partir de textos escrits en l'anglès utilitzat a Nova Zelanda. El corpus es desenvolupà per a fer-lo comparable als corpus de LOB i Brown descrits anteriorment (seccions \ref{sec:corpus_brown} i \ref{sec:corpus_lob}). Conté també 500 textos de diferents categories, d'unes 2000 paraules cada text i al voltant d'un milió de paraules diferents en tot el corpus. La taula \ref{tab:corpus_lm} conté un resum dels tres corpus utilitzats. Aquest corpus fou utilitzat per a construir el model del llenguatge utilitzat en els experiments.

\begin{table}
\centering
\begin{tabular}{|c|l|c|c|c|}
\hline
Categoria & Descripció & Brown & LOB & Wgton\\
\hline
A & Premsa: reportatges & 44 & 44 & 44\\
B & Premsa: editorial & 27 & 27 & 27\\
C & Premsa: ressenyes & 17 & 17 & 17\\
D & Religió & 17 & 17 & 17\\
E & Habilitats, oficis i aficions & 36 & 38 & 38\\
F & Tradició popular & 48 & 44 & 44\\
G & Belles lletres, biografia, assajos & 75 & 77 & 77\\
H & Miscel·lània & 30 & 30 & 30\\
J & Escrits científics & 80 & 80 & 80\\
K & Ficció general & 29 & 29 & 29\\
L & Ficció de misteri i policiaca & 24 & 24 & 24\\
M & Ciència Ficció & 6 & 6 & 6\\
N & Ficció d'aventures i de l'oest & 29 & 29 & 29\\
P & Històries d'amor i romàntiques & 29 & 29 & 29\\
R & Humor & 9 & 9 & 9\\
\hline
Total & & 500 & 500 & 500\\
\hline
\end{tabular}
\caption{Resum dels textos continguts en els corpus Brown, LOB i Wellington.}\label{tab:corpus_lm}
\end{table}
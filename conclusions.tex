\chapter{Conclusions}
\label{cap:con}

La taula \ref{tab:summary} situa els resultats obtinguts en aquest treball en comparació amb resultats obtinguts prèviament en la mateixa tasca de reconeixement.\\

\begin{table}
\begin{center}
\begin{tabular}{|l|c|c|}
\hline
& Validació & Test \\\hline\hline
Bertolami et al. \cite{bertolami2008hidden} & \textbf{30.98} & \textbf{35.52} \\\hline
España Boquera et al. \cite{espana2011improving} & 32.80 & 38.80  \\\hline
Seg. Supervisada & 32.99 & 40.08 \\\hline
Seg. Heurística & 38.74 & 45.54 \\\hline
\end{tabular}
\caption{Comparació del WER de les alternatives estudiades amb altres publicacions.}\label{tab:summary}
\end{center}
\end{table}

En aquesta taula s'observa que els resultats obtinguts a partir del sistema que empra una segmentació del cos central basada en aprenentatge supervisat no correspon als resultats obtinguts en la publicació. Això és degut a que, a pesar d'intentar replicar els experiments amb el major detall possible i comptar amb la col·laboració dels autors de l'article, hi ha nombrosos detalls sobre l'ex\-pe\-ri\-men\-ta\-ció publicada que eren desconeguts i/o no van poder ser replicats. D'ací aquesta petita diferència en les mesures obtingudes.\\

Més important encara, s'observa que l'utilització d'una segmentació del cos central del text manuscrit utilitzant aprenentatge supervisat redueix un $11.99\%$ el WER respecte a l'alternativa tradicional basada en un enfocament heurístic, en la tasca de reconeixement del corpus IAMDB, un important corpus utilitzat a l'hora de mesurar els errors en les transcripcions dels sistemes de reconeixement automàtic de text manuscrit.\\

Queda per tant comprovada la hipòtesi que es presentava al descriure les diferències entre els dos enfocaments per a la segmentació del cos central del text manuscrit: l'ús d'una tècnica basada en aprenentatge supervisat millora significativament la segmentació heurística. Aquesta millora a sigut quantificada satisfactoriament a pesar de les dificultats a l'hora de reproduir els detalls de les experimentacions publicades amb anterioritat i sols queda per tant fer front a la qüestió de si aquesta millora compensa el cost econòmic i temporal d'utilitzar aquesta tècnica basada en aprenentatge supervisat per a sistemes reals.